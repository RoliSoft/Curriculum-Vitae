\documentclass[11pt,a4paper]{article}

\usepackage[utf8x]{inputenc}
\usepackage[english]{babel}
\usepackage[margin=1in,includefoot]{geometry}

\usepackage{fancyhdr}
\usepackage{indentfirst}
\usepackage{fontawesome}

\usepackage{hyperref}
\hypersetup{pdfborder={0 0 0}}

\usepackage[usenames,dvipsnames]{xcolor}
\colorlet{LightGray}{Black!20!}
\definecolor{LightBlue}{rgb}{0.22,0.45,0.70}
\definecolor{DarkGray}{rgb}{0.45,0.45,0.45}
\definecolor{DarkGreen}{rgb}{0.35,0.70,0.30}

\usepackage{sectsty}
\sectionfont{\color{LightBlue}}
\subsectionfont{\color{DarkGreen}}

\usepackage{enumitem}
\setlistdepth{9}
\setlist[itemize,1]{label=\large{$\bullet$}}
\setlist[itemize,2]{label=\large{$\circ$}}
\setlist[itemize,3]{label=\tiny{$\blacksquare$}}
\setlist[itemize,4]{label=\large{$\bullet$}}
\setlist[itemize,5]{label=\large{$\circ$}}
\setlist[itemize,6]{label=\tiny{$\blacksquare$}}
\setlist[itemize,7]{label=\large{$\bullet$}}
\setlist[itemize,8]{label=\large{$\circ$}}
\setlist[itemize,9]{label=\tiny{$\blacksquare$}}
\renewlist{itemize}{itemize}{9}

\usepackage[letterspace=150]{microtype}

\renewcommand{\familydefault}{\sfdefault}

\usepackage{amsmath}
\usepackage{amsfonts}
\usepackage{amssymb}

%\overfullrule=1mm

\author{Roland Bogosi}
\title{Curriculum Vitae}

\begin{document}

	\begin{center}
	\textsc{\Huge Roland Bogosi}\\
	\vspace{-2mm}
	\rule{300pt\color{LightGray}}{0.5px}\\
	\vspace{0.5mm}
	{\lsstyle\color{DarkGray} \faGlobe\ Unit 7537, PO Box 6945, London, W1A 6US, UK\\
	\faPhone\ \href{tel:+442034111758}{\color{DarkGray} +44 20 3411 1758} {\color{LightGray}$\bullet$} \faPhone\ \href{tel:+12243667654}{\color{DarkGray} +1 224 366 7654}\\
	\faEnvelopeO\ \href{mailto:root@rolisoft.net}{\color{DarkGray} root@rolisoft.net}}
	\end{center}

\section*{Objective}
	Seeking a position to utilize my skills and abilities in the \textbf{Information Technology} and \textbf{Electrical Engineering Industry} that offers professional growth while being \textbf{resourceful}, \textbf{innovative} and \textbf{flexible}.

\section*{Education}
	\begin{itemize}
	\item	{\large\textbf{B.Sc. Computer Engineering}}\\
			Sapientia EMTE, Faculty of Technical and Human Sciences, Târgu-Mureș\\
			\textit{September 2012 – July 2016 (currently attending)}
	\end{itemize}

\section*{Experience}
	\begin{itemize}
	\item	{\large\textbf{BrightAlien Ltd}}\\
			\textbf{Senior CTO, Part-Time Full-Stack Developer}\\
			Job responsibilities include the full-stack development of 9 websites plus an internal dashboard; the installation with continued proactive monitoring and administration of 4 *nix-based and 1 Windows-based production servers; miscellaneous devops responsibilities.\\
			\textit{December 2013 – Present}
	\item	{\large\textbf{Lynx Solutions}}\\
			\textbf{Android Developer}\\
			Job responsibilities included the creation of a 30-page document describing the application to be developed and its specifications; the single-handed development of the documented Android-based application with UI accommodating both smartphones and tablets alike (through the use of fragments and multiple layouts with constraints); overseeing fellow interns through their journey of implementing the documented REST API.\\
			\textit{July 2014 (one-month internship)}
	\end{itemize}

\section*{Languages}
	\begin{itemize}
		\item	{\large\textbf{English}}: Full professional proficiency
		\item	{\large\textbf{Hungarian, Romanian}}: Native or bilingual proficiency
	\end{itemize}

\newpage

\section*{Technical Skills}
	\begin{itemize}
	\item	\textbf{11 years} of experience in \textbf{HTML}, \textbf{CSS}, \textbf{JavaScript}, \textbf{PHP} and \textbf{MySQL}.
		\begin{itemize}
		\item	First exposure to the world of programming via PHP at the age of \textbf{11}.
		\item	Published first dynamic webpage written from scratch at the age of \textbf{12}.
		\item	Received first \$100 check from \textbf{Google AdSense} at the age of \textbf{14}.
		\item	Up-to-date knowledge of \textbf{HTML 5}, \textbf{CSS 3}, \textbf{PHP 5}, \textbf{JavaScript}, and \textbf{SQL} as of today.
		\item	Extensive knowledge in the use and production of \textbf{REST APIs}, \textbf{OAuth}, \textbf{OpenID} and \textbf{SOAP}, also having experience in \textbf{data-interchange formats}, namely: \textbf{JSON}, \textbf{XML}, \textbf{Google Protocol Buffers}, \textbf{BSON}, and various others.
		\item	Knowledge of various \textbf{SQL flavors}, in order of skill level: \textbf{MySQL}, \textbf{SQLite}, \textbf{Oracle}, \textbf{PostgreSQL} and \textbf{Microsoft SQL Server}.
		\item	Extensive knowledge of \textbf{optimization and caching mechanisms}, including the optimization of \textbf{relational databases} (\textbf{query plans}, \textbf{indices}, \textbf{complexities}, \ldots), \textbf{key-value stores} (such as \textbf{Redis} and \textbf{memcached}). \textbf{Multi-tiered caching systems}, utilizing the aforementioned technologies, with a fallback to file-based \textbf{LRU caches}. Brief exposure to \textbf{document stores}, namely \textbf{MongoDB} and \textbf{CouchDB}.
		\item	Extensive exposure to \textbf{dedicated search servers}, namely \textbf{ElasticSearch} and \textbf{Sphinx}, including both their programmatic use and proper server configuration. 
		\item	Experience with \textbf{jQuery}, \textbf{semantic web} and \textbf{dynamic webpages}, utilizing \textbf{AJAX}, \textbf{HTML5 History API}, and so on.
		\item	Experience with \textbf{Bootstrap}, and \textbf{grid-based fluid} and \textbf{responsive web design}.
		\item	Familiarity with \textbf{Model-View-Controller} (\textbf{MVC}) architectures.
		\item	Extensive knowledge of \textbf{Ajax} and \textbf{dynamic web-apps}, and the utilization of \textbf{JavaScript "MVW"} frameworks, namely \textbf{Angular.js}.
		\item	Extensive experience with \textbf{test-driven development} via \textbf{PHPUnit}, and the use of various development tools, such as \textbf{performance profilers} and \textbf{remote debuggers}, namely \textbf{XDebug}.
		\item	Experience with both \textbf{horizontal} and \textbf{vertical scaling} of both \textbf{databases} and \textbf{codes}.
		\item	Familiarity with the integration and implementation of various \textbf{payment gateway processors}, such as \textbf{PayPal IPN}, \textbf{Stripe}, and \textbf{Bitcoin RPC}.
		\item	Intricate knowledge on the \textbf{security} front, and up-to-date on the \textbf{0-day scene}. While programming, I have a \textbf{security- and optimization-focused} mindset. While browsing, I have a tendency to alert server owners of potential exploits on the server. 
		\item	Currently running \textbf{4 active websites}, with a total of \textbf{20 million page views per month}. The sites were written from scratch by me, and the servers on which they're running on were also configured by me. They are deployed to \textbf{cloud-provider}, namely \textbf{DigitalOcean}, and are being served in a \textbf{load-balanced} and \textbf{dual-stacked environment}, through a $3^{rd}$-party Content Distribution Network, \textbf{CloudFlare}.
		\end{itemize}
	\item	\textbf{9 years} of experience in \textbf{.NET}/\textbf{C\#}.
		\begin{itemize}
		\item	First exposure to object-oriented programming in \textbf{Visual Basic .NET}, then \textbf{C\#}, via the .NET framework at the age of \textbf{13}.
		\item	Published first \textbf{desktop application}, which implemented an \textit{original idea}, at the age of \textbf{13}, which utilized \textbf{databases}, \textbf{regular expressions}, \textbf{networking} and \textbf{screen-scraping}. Application was published to \textbf{Softpedia}, where it was accepted and reviewed by their staff, leading to a few hundred downloads during its lifetime. 
		\item	Use of \textbf{neural networks}, in order to train a software to \textbf{recognize} numbers and letters that appear on a \textbf{webcam}-provided image in real-time, at the age of \textbf{16}.
		\item	Published first \textbf{open-source application} at the age of \textbf{16}, which then went \textbf{commercial} at the age of \textbf{18}, by integrating serial numbers generated and validated using \textbf{RSA public-private key encryption}, which were released by a \textbf{PHP script} called by either \textbf{PayPal Instant Payment Notification API} or the \textbf{Bitcoin API}.
		\item	Experience with \textbf{NSIS}, \textbf{InnoSetup}, and \textbf{WiX}, including scripting of custom methods.
		\item	Extensive knowledge of \textbf{Test-Driven Development} via leading \textbf{unit test libraries}, such as \textbf{NUnit} and \textbf{MSTest}.
		\item	Extensive experience with various development tools, used to:
			\begin{itemize}
			\item	profile and \textbf{analyze unit test coverage}, such as \textbf{dotCover};
			\item	profile, detect and \textbf{mitigate memory leaks}, such as \textbf{ANTS Memory Profiler} and \textbf{dotMemory};
			\item	profile, detect and \textbf{mitigate performance bottlenecks}, such as \textbf{ANTS Performance Profiler} and \textbf{dotTrace}.
			\end{itemize}
		\item	Various short-term hobby projects along the time driven by the desire to learn and experiment in various fields, thus expanding my experience and increasing my knowledge of various algorithms, frameworks, and my programming ability as a whole.
		\end{itemize}
	\item	\textbf{7 years} of experience in \textbf{Unix-like systems} (\textbf{Linux}, \textbf{BSD})
		\begin{itemize}
		\item	Installed first \textbf{Linux distribution} at the age of \textbf{14}.
		\item	Flashed first router with \textbf{OpenWRT} and \textbf{DD-WRT} at the age of \textbf{15}.
		\item	Complied first PHP tarball \textbf{used in production} on \textbf{unmanaged VM} at the age of \textbf{15}. 
		\item	Installed first DIY-type distributions, \textbf{ArchLinux}, \textbf{Gentoo} and then ultimately \textbf{Linux from Scratch}, at the age of \textbf{16}.
		\item	Configured first server (\textbf{LAMP + exim}), from \textbf{scratch}, on an \textbf{unmanaged VM}, still used in production today, at the age of \textbf{16}.
		\item	Extensive knowledge of the \textbf{Linux} ecosystem as of today. Personally owning and administering \textbf{3 production and staging server}, while providing support for others on an incident-response basis.
		\item	Early adopter on multiple fields, including \textbf{IPv6}, having been participated in the \textbf{private beta} of the IPv6 deployment of both \textbf{RDS\&RCS} and \textbf{Dreamhost}. The configuration and use of \textbf{IPv6 networking} was the primary reason to stay with \textbf{OpenWRT}-flashed routers in the first place, as it was not yet available in consumer firmware due to the low technology adoption rate at the time.
		\item	Tendency to \textbf{automate tedious processes}, by \textbf{scripting them} in the appropriate environment. (Such as \textbf{shell scripts} in \textbf{Unix} and \textbf{Unix-like systems}.)
		\item	\textbf{Strong Bash scripting} experience, including \textbf{heavy terminal usage}, and heavy knowledge of the \textbf{BSD} and \textbf{GNU userland tools}.
		\item	Ability to configure, administer, update and support \textbf{Debian} and \textbf{CentOS}-based systems for various purposes:
			\begin{itemize}
			\item	Web servers: \textbf{nginx}, \textbf{lighttpd}, \textbf{Apache}, with:
				\begin{itemize}
				\item	Fast and secure configurations, including, but not limited to:
					\begin{itemize}
					\item	\textbf{Caching} and \textbf{microcaching} techniques, including on-the-fly optimization with \textbf{Google PageSpeed} modules, and \textbf{client-side} caching, such as serving files with a correct cache-expiry header based on their \textbf{MIME-type}.
					\item	\textbf{Web Application Firewall} security modules, such as \textbf{mod\_security} and \textbf{mod\_evasive}.
					\item	Load-balanced environments, including:
						\begin{itemize}
						\item	\textbf{nginx} frontend load-balancing to multiple backends
						\item	\textbf{Varnish}/\textbf{nginx} cache in front of \textbf{Apache}
						\item	$3^{rd}$-party CDN integration (\textbf{CloudFlare}, \textbf{Incapsula}, etc)
						\end{itemize}
					\end{itemize}
				\item	\textbf{PHP} and other languages that have a \textbf{FastCGI} frontend or otherwise a module for the server software in question.
					\begin{itemize}
					\item	\textbf{HHVM}-backend as an alternative for \textbf{PHP} on \textbf{high-load sites}.
					\end{itemize}
				\item	\textbf{SSL/TLS secure server} configuration, including:
					\begin{itemize}
					\item	Preparation and deployment of either a trusted or a self-signed certificate.
					\item	Correct configuration of the webserver in question to mitigate CRIME, BEAST, Heartbleed and various other known threats.
					\item	Use of \textbf{HSTS headers} and manual HTTPS redirection where appropriate.
					\item	Use of \textbf{SNI}, which allows the use of multiple certificates on one IP, when necessary in a shared environment.
					\end{itemize}
				\end{itemize}
			\item	Email servers: \textbf{exim}, \textbf{Postfix}, \textbf{Dovecot}, with:
				\begin{itemize}
				\item	Set-up of the domain names to receive/send emails from/to with the correct anti-spam and authentication methods being used today:
					\begin{itemize}
					\item	\textbf{Sender Policy Framework} (\textbf{SPF} records, in \textbf{TXT} and \textbf{DNS\#99})
					\item	\textbf{DomainKeys Identified Mail} (\textbf{DKIM} DNS records and signing at the mailer daemon level)
					\item	\textbf{Domain-based Message Authentication, Reporting and Conformance} (\textbf{DMARC} DNS records, and interpretation of the incoming reports)
					\item	\textbf{Author Domain Signing Practices} (\textbf{ADSP} DNS records)
					\end{itemize}
				\end{itemize}
			\item	Database servers and various stores: \textbf{MySQL} (and forks, namely \textbf{MariaDB} and \textbf{PerconaDB}), \textbf{PostgreSQL}, \textbf{Redis}, \textbf{CouchDB}, \textbf{memcached}, etc.
			\item	Search servers: \textbf{ElasticSearch}, \textbf{Sphinx}, with optional content synchronization.
			\item	Proxies and VPN servers: \textbf{OpenVPN}, \textbf{L2TP/IPSec}, \textbf{PPTP} protocols, through various daemons which the selected distribution recommends, including multifunctional servers, such as \textbf{SoftEther}.
			\item	File servers: \textbf{Samba}, \textbf{NFS}
			\item	Authentication servers: \textbf{FreeRADIUS}, \textbf{OpenLDAP}
			\item	Type-1 hypervisors: \textbf{VMware ESXi}, \textbf{Microsoft Hyper-V}
			\item	Experience with \textbf{hidden services}, including configuring the \textbf{middleware} and securing the servers behind it not to leak personally-identifiable information:
				\begin{itemize}
				\item	\textbf{Onion-routed hidden services} (\textbf{.onion}) through \textbf{Tor}
				\item	\textbf{Garlic-routed EepSites} (\textbf{.i2p}) through \textbf{I2P}
				\end{itemize}
			\item	Exposure to \textbf{meshnets}, such as the \textbf{Hyperboria network} on the \textbf{CJDNS}.
			\item	Experience with \textbf{DNS configuration}, including \textbf{dynamic DNS setups}, \textbf{anti-spam records}, \textbf{wildcard setups}, and \textbf{load-based dynamic setups}.
			\end{itemize}
		\item	Extensive experience with \textbf{cloud-service providers}, namely \textbf{Amazon AWS}, \textbf{Linode}, and \textbf{DigitalOcean}.
		\end{itemize}
	\item	\textbf{Exposure through personal and open-source projects}, to various languages, technologies, techniques and practices:
		\begin{itemize}
		\item	Exposure to the inner workings of \textbf{Android}, via the use of open-source after-market firmware and side-projects that focus on the system- or algorithm-level:
			\begin{itemize}
			\item	\textbf{HTTP networking}, \textbf{location-services}, \textbf{camera}, \textbf{SuperSU toolkit}, and various other \textbf{Android APIs}, some accessed through \textbf{reflection}, in a side-project that accepts commands from an \textbf{XMPP} server and carries them out. Gateway is written in \textbf{Python} and deployed on \textbf{Google AppEngine}.
			\item	\textbf{TCP networking and UDP broadcast system}, in a side-project that uses the device's \textbf{gyroscope} and/or \textbf{accelerometer} and sends data through the sockets, triggering, in this case, the movement of the mouse on the connected PC. The UDP broadcast system was used to discover TCP servers, in order to ease the use of the software, as writing IP addresses is tiresome. Such \textbf{consumer-oriented features} are to be expected from me.
			\end{itemize}
		\item	First exposure to \textbf{autonomous web-services} that rely on user-contributed data was published at the age of \textbf{17}. Aforementioned web-service uses \textbf{PHP}/\textbf{MySQL} and is a \textbf{recommendation system}, which provides \textbf{tailored recommendations} based on a user's preference list.
			\begin{itemize}
			\item	The recommendation engine was integrated through its Web API, by $3^{rd}$-party developers, into \textbf{MythTV} as a $3^{rd}$-party user-contributed plugin.
			\end{itemize}
		\item	Exposure and experiments with massively parallel systems, such as \textbf{OpenCL} and \textbf{nVidia}'s \textbf{CUDA}, which are \textbf{GPU-backed} languages.
		\item	Additional exposure (besides the aforementioned languages) through side-projects with the following languages, in order of fluency:
			\begin{itemize}
			\item	\textbf{Python} (\textbf{Django} and \textbf{Google AppEngine})
			\item	\textbf{Ruby} (\textbf{Ruby on Rails})
			\item	\textbf{PowerShell}
			\end{itemize}
		\end{itemize}
	\item	Gained deeper insights into various fields pertaining to \textbf{Computer Science} and \textbf{Electrical Engineering}, of whose knowledge were previously vague and/or fragmented:
		\begin{itemize}
		\item	Use of \textbf{Matlab} to solve problems, analyze data, draw in 2D and 3D space, and perform simulations.
		\item	Knowledge of programming techniques, structures, algorithms and graph theory.
		\item	Furthered knowledge in the field of relational databases, including relational algebra, database normalization, and so on.
		\item	Intricate knowledge of the UNIX operating system, and general theories belonging to Operating Systems and computers in general.
		\item	Introduction to the world of \textbf{Logical} and \textbf{Functional programming}, beginning from the implementation of various algorithms in a recursive way, until the implementation of more complex applications with \textbf{GUI}.
		\item	\textbf{Analogue} and \textbf{Digital integrated circuits}, \textbf{Boole logic}, \textbf{Automata theory}, \textbf{finite-state automaton implementations} using analogue circuit elements, binary systems, decoders, (de-)multiplexers, mathematical operations, amplifiers, operational amplifiers, counters, mono-/bi-stables, registers, and memory types.
		\item	\textbf{FPGA} programming, including implementation of state-machines using either or both \textbf{VHDL programming} and/or \textbf{schematic design} of circuit elements and gates.
		\item	\textbf{Microcontrollers}, such as \textbf{PIC} and \textbf{Atmel} programmed in \textbf{ANSI C}. Intricate knowledge of the inner workings of a microprocessor thanks to \textbf{Microcontroller Design}, \textbf{Computer Architecture} and \textbf{Assembly Language} classes.
		\item	Built a fully working \textbf{CPU} in \textbf{FPGA} from scratch for a class assignment.
		\item	Learned the advanced parts of \textbf{Artificial Intelligence} via the courses.
		\item	Furthered knowledge in the field of networking by embarking on a journey to study the intricacies of various \textbf{networking protocols} as an extracurricular activity, by writing a WiFi packet capture software that also analyzes the received data.
		\end{itemize}
	\item	Gave multiple presentations in classes as an extracurricular activity:
		\begin{itemize}
		\item	\textbf{Facebook API and Query Language} on Database class, wherein I presented the practical uses of a database in real-life scenarios via Facebook's FQL interface.
		\item	\textbf{The Deep Web} on Software Engineering class, where I talked about the general design of anonymous networks, including the networking and encryption parts. The presentation also included a general introduction to the deep web as we use it today on Tor, such as how anonymous payments are made, and then ended on a short live demo of browsing Tor.
		\item	\textbf{Peer-to-Peer Systems} on Distributed Networks class, where I delved into the history of Peer-to-Peer networks, and their general structures, ending with a few case studies of important and popular P2P networks.
		\end{itemize}
	\item	Worked on multiple projects as \textbf{extracurricular school projects}, these are listed at the bottom of the CV.
	\item	Additionally, strong self-taught knowledge in the following fields:
		\begin{itemize}
		\item	\textbf{Cross-platform programming}
			\begin{itemize}
			\item	\textbf{Strong C++} experience
				\begin{itemize}
				\item	Libraries: \textbf{STL}, \textbf{boost}
				\item	Widget toolkits: \textbf{Qt}, \textbf{MFC}
				\item	Up-to-date with the standards (\textbf{C++14})
				\end{itemize}
			\item	Java
				\begin{itemize}
				\item	\textbf{Android} on mobile
				\item	\textbf{Swing GUI} on desktops
				\end{itemize}
			\end{itemize}
		\item	\textbf{Cryptography}
			\begin{itemize}
			\item	\textbf{Block ciphers} (e.g. \textbf{AES}, \textbf{Serpent}, \ldots)
			\item	\textbf{Stream ciphers} (e.g. \textbf{RC4}, \textbf{Salsa20}, \ldots)
			\item	\textbf{Public-private key cryptography} (\textbf{RSA})
			\item	Correct use of \textbf{block cipher mode of operation} (e.g. \textbf{CBC})
			\item	\textbf{One-way hashing algorithms} (e.g. \textbf{SHA-1})
			\item	\textbf{Fuzzy-hashing algorithms} (e.g. \textbf{ssdeep})
			\item	\textbf{Key-exchange algorithms} (e.g. \textbf{Diffie-Hellman})
			\item	\textbf{One-Time Password solutions} (e.g. \textbf{RFC6238})
			\item	Studied the use of cryptography in existing systems, such as \textbf{SSL/TLS}, email encryption/authentication via \textbf{PGP} and \textbf{S/MIME}, various secure routing systems, such as \textbf{Onion-routing}.
			\item	Up-to-date knowledge on storing and using information securely. (Such as the use of \textbf{bcrypt} for storing passwords.)
			\end{itemize}
		\item	\textbf{Networking}
			\begin{itemize}
			\item	\textbf{TCP stream} and \textbf{UDP datagram sockets}
			\item	\textbf{RUDP} (\textbf{Reliable UDP}) implementations
			\item	\textbf{Secure sockets}, including, but not limited to: \textbf{end-point authentication}, \textbf{encryption of data stream}, and \textbf{insurance of data authenticity}.
			\item	\textbf{IPv4-specific} technologies, such as \textbf{broadcasts}, \textbf{ARP}, etc.
			\item	\textbf{IPv6-specific} technologies, such as \textbf{multicasts}, etc.
			\item	\textbf{IPv4} and \textbf{IPv6 dual-stacking}
			\item	\textbf{Subnetting}
			\item	\textbf{Tunneling} (\textbf{Layer-2/3} via \textbf{VPN} setups)
			\item	\textbf{UPnP} port-forwarding via \textbf{IGD API} endpoints
			\item	\textbf{NAT} traversal, including \textbf{classification} (full-cone, restricted-cone, symmetrical) and potentially \textbf{hole-punching}, through the use of \textbf{STUN}/\textbf{TURN}/\textbf{ICE} servers
			\item	\textbf{DHCP servers setups} (e.g. \textbf{TFTP} address for \textbf{PXE netboots})
			\item	\textbf{Strong knowledge} of \textbf{Layer-7 protocols}, including, but not limited to:
				\begin{itemize}
				\item	\textbf{HTTP}, \textbf{SPDY}
				\item	\textbf{SMTP}; \textbf{IMAP}, \textbf{POP3}
				\item	\textbf{SSH}, \textbf{Telnet}
				\item	\textbf{SFTP}, \textbf{FTP}
				\item	\textbf{NFS}, \textbf{Samba}
				\item	\textbf{SIP}, \textbf{XMPP}, \textbf{IRC}
				\item	\textbf{NTP}; \textbf{DNS}
				\end{itemize}
			\end{itemize}
		\item	\textbf{Penetration Testing}
			\begin{itemize}
			\item	Up-to-date with \textbf{security bulletins}.
			\item	Knowledge of \textbf{Cross-Site Scripting}, \textbf{SQL Injection}, \textbf{Cross-Site Request Forgery} and similar techniques from a very young age, around \textbf{11} or so.
			\item	Familiarity with the tools used in the industry of \textbf{pentesting}, such as \textbf{Metasploit}, \textbf{Acunetix}, \textbf{BackTrack}, etc.
			\item	Familiarity with \textbf{Web Application Firewalls} and \textbf{Intrusion Detection Systems}, techniques to bypass them, and to strengthen them for different purposes.
			\item	Notified \textbf{Sapientia} of being \textbf{vulnerable to OpenSSL's heartbleed bug}, approximately 5 hours after the public disclosure of the bug. Personal servers have been patched within \textbf{3 hours after public disclosure}.
			\end{itemize}
		\item	\textbf{Reverse Engineering}
			\begin{itemize}
			\item	Experience with \textbf{disassembling}, \textbf{modifying} and \textbf{reassembling MSIL/CIL} and \textbf{Java bytecode}.
			\item	Familiarity with \textbf{control flow manipulation} in \textbf{x86(-64) ASM}.
			\item	Familiarity with the tools used in \textbf{reverse engineering}, such as:
				\begin{itemize}
				\item	Debuggers: \textbf{WinDbg}, \textbf{OllyDbg}; originally trained with \textbf{SoftICE}.
				\item	Disassemblers/Decompilers: \textbf{IDA Pro}, \textbf{Reflector}, \textbf{JD}, amongst others.
				\item	Various file format/PE analysis tools. (e.g. \textbf{NTCore's CFF})
				\item	Popular packers and their unpacker equivalents. (e.g. \textbf{ASPack}, \textbf{UPX})
				\end{itemize}
			\end{itemize}
		\item	\textbf{Screen-Scraping}
			\begin{itemize}
			\item	Strong knowledge of \textbf{regular expressions} and \textbf{XPath}.
			\item	Complicated setups involving scripts that circumvent anti-screen-scraping measures, even to the point of \textbf{OCR}-ing the Captcha, when it is weak enough, otherwise using the human-powered services available on the markets.
			\end{itemize}
		\item	\textbf{Search Engine Optimization}
			\begin{itemize}
			\item	\textbf{Google Panda}-tailored optimizations that have proven their legitimacy throughout the various websites I operate.
			\end{itemize}
		\item	\textbf{Version Control Systems}
			\begin{itemize}
			\item	Extensive use of \textbf{git} nowadays.
			\item	Previously used \textbf{SVN}.
			\item	Familiarity with other VCSs, such as \textbf{Mercurial} and \textbf{CVS}.
			\end{itemize}
		\item	\textbf{Continuous Integration}
			\begin{itemize}
				\item	Experience with CI through usage in open-source projects.
				\item	Familiarity with the set-up and usage of \textbf{Jenkins} and \textbf{Travis CI}.
			\end{itemize}
		\end{itemize}
	\item	\textbf{Self-teaching} is an important part of my lifestyle. I try to keep up with the ever-evolving technologies of today.
		\begin{itemize}
		\item	I constantly \textbf{try out new languages}, \textbf{new technologies} and \textbf{new practices}. I constantly go back and \textbf{re-do old projects}, with a twist, and I challenge myself to do it much better this time, by setting much higher goals.
		\item	I read \textbf{research papers}, \textbf{0-day bulletins} and other sources of information that let me be ahead of the competition.
		\item	I am not afraid to take initiative, and to color outside of the lines. I don't mind to get my hands dirty to try something out, let it be DoS-ing my own server in order to try out if an optimization technique or security practice did indeed work.
		\item	I also watch \textbf{conference videos} (such as \textbf{DefCon}, \textbf{Black Hat}, etc) and \textbf{tech-talks} (such as \textbf{Microsoft's GoingNative}, and much more) in order to get an edge in their specific fields.
		\item	Whenever writing code, I focus on:
			\begin{itemize}
			\item	\textbf{simplicity},
			\item	\textbf{code-maintainability},
			\item	\textbf{optimization}, and
			\item	\textbf{security}.
			\end{itemize}
		\end{itemize}
	\item	\textbf{Communication skills} gained through regular business and social interaction with clients and fellow developers.
	\item	\textbf{Presentation skills} acquired by frequently giving presentations about various topics to varying audiences.
	\item	\textbf{Organizational}/\textbf{Managerial skills} accumulated over time via superfluous scheduling and prioritizing to meet business and educational deadlines.
	\end{itemize}

\section*{Personal Commercial Projects}
	\subsection*{RS TV Show Tracker}

		An \textbf{open-source} application which was born out of the need for the features it currently offers, as there were no alternative solutions for them at the time. To date, no software has so many features in this category, and as a result, under the four years it's been actively developed, its popularity has grown exponentially. Today, I single-handedly maintain it and push updates to hundreds of thousands of users on a monthly basis.
		
		The application was developed in \textbf{C\#} with an interface in \textbf{WPF}, and was kept constantly up-to-date with the newer technologies that have been released during its development phase.

		There are, as of writing this in May 2015, \textbf{135,700 active daily users} of the application, with the number of installations reaching into the \textbf{millions}. Development started in \textbf{February 2010}.

	\subsection*{AlienSubtitles.com}

		\textbf{Full-stack development} of the website and continued maintenance, including initial \textbf{devops responsibilities} (such as cloud server deployment on DigitalOcean) and continued \textbf{proactive monitoring} and administration of the web, database and load-balancer servers.

		The website was developed in \textbf{PHP} (\textbf{HHVM}) backed by \textbf{MySQL} servers in \textbf{multi-master replication} configuration and \textbf{ElasticSearch} instances for the search feature.

		At its peak, the site had \textbf{15 million unique users}, who were served by 4 geo-located and load-balanced front-end servers, attaining \textbf{100\% uptime} and \textbf{under 10 millisecond response times} for all users throughout peak times. Development started in \textbf{December 2013}.

\section*{Extracurricular School Projects}
	\begin{itemize}
	\item	\textbf{Streaming and Processing Sensor Data from Android Devices in Realtime} (\textit{Object-Oriented Programming})
		\begin{itemize}
		\item	This is a two-component application, which when used together can move the mouse on the computer of the user by moving a smartphone in the air.
		\item	AirMouse-Java-Server: Control the mouse of your PC or laptop by physically moving your phone in the air! Once an Android client connects to this Server, the streamed sensor data will be translated into up-down-left-right movements, which will be reflected upon the mouse on your computer.
		\item	AirMouse-Android-Client: An Android application, which streams accelerometer and/or gyroscope sensor data to a server application over the Internet or on the local network, where UDP broadcast-based service discovery is available for a faster server listing.
		\end{itemize}
	\item	\textbf{On-Demand Task Loader and Executor Unix Daemon} (\textit{Advanced Programming Techniques})
		\begin{itemize}
		\item	Project required the development of a UNIX daemon in C++ which would load and execute dynamically-linked libraries (.so) dropped into its monitored designated directory. It would also stop its execution and unload it if the .so file is removed, and more importantly re-initialize the library and restart the execution if the .so file is modified during its execution.
		\item	The library file had to be as independent as possible from the daemon application, which was successfully accomplished by giving the daemon application excellent customizability. The library file does not need to link with any component of the daemon.
		\item	The technologies used in the project included pthreads, inotify and signals.
		\end{itemize}
	\item	\textbf{Facebook FQL Query Tool and SQL Schema/Data Exporter} (\textit{Databases})
		\begin{itemize}
		\item	This is a two-part project, which first interfaced with Facebook's FQL API and provided the students with a familiar and easy-to-use interface to test their SQL skills on a real database, consisting of their Facebook data.
		\item	The second part of the project allows the users to export the data of their query, including an auto-generated schema, so they can import it into their RDBMS of choice and perform further analysis on the data.
		\end{itemize}
	\item	\textbf{Network Analysis and Protocol Dissection of Sniffed WiFi Traffic} (\textit{Distributed Networks})
		\begin{itemize}
		\item	This modular C++ project uses libpcap and the AirPcap Nx USB dongle, which is capable of sniffing the unencrypted traffic of a specified WiFi channel.
		\item	The raw bytestream captured WiFi traffic is then analyzed using the implemented dissectors, which are modules in the application capable of handling various protocols pertaining to various levels of the OSI layer.
		\item	The point of this project is educational, to gain insights into the inner workings of various protocols as they are implemented and used on the wire. (Or, in our case, in the air.)
		\end{itemize}
	\item	\textbf{An Untitled Minecraft-like 3D Multiplayer Game} (\textit{Software Engineering})
		\begin{itemize}
		\item	This project was a special group project, consisting of 5 students, who were required to do the planning, documentation and development of a 3D game (developed in C++ with OpenGL) that allowed players to play a Minecraft-like Creative game session over the internet.
		\item	My responsibilities were the planning, documentation and development of the networking part of the game, which, besides the protocol and the network API, also included the synchronization of player/map/game states and the insurance of a smooth gameplay. (For example, elimination of the rubber-banding effect to the best of my abilities.)
		\end{itemize}
	\item	\textbf{Assembly Compiler for Course-made CPU on FPGA} (\textit{Computer Architecture})
		\begin{itemize}
		\item	As part of the Computer Architecture course, a fully working CPU was built from scratch on an FPGA. For testing, we were given .coe files in order to evaluate that our CPU is working properly, and writing an Assembler that would produce these was not part of the course.
		\item	This project assembles .coe files to be used with Xilinx ISE, out of .asm files which use a custom Intel-like syntax, with added MASM-like syntax sugar, such as \#if, \#while and \#for macros. The assembler was written in C++.
		\end{itemize}
	\item	\textbf{Sentimental Analysis of Text on Social WebSites} (\textit{Databases II})
		\begin{itemize}
		\item	The first stage of the project was for it to be able to determine the positive, negative and objective connotations of a given text and score it on a scale.
		\item	Further advanced features of the project include the ability to determine the attitude of your friends from social sites (Facebook, Twitter) towards a given item/product.
		\end{itemize}
	\end{itemize}
	
	{\color{DarkGray} The latest revision of my CV can be downloaded from}
	{\color{LightBlue} \url{https://rolisoft.net/cv.pdf}

\end{document}