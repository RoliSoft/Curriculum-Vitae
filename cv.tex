\documentclass[11pt,a4paper,sans]{moderncv}

\moderncvstyle[shortrules]{banking}
\moderncvcolor{blue}

\usepackage[utf8]{inputenc}
\usepackage[scale=0.8]{geometry}
\usepackage{combelow}

% create new hlink command; this is needed, since \httplink only supports http:// links.
% also overwrite \httplink to point to HTTPS addresses.
\makeatletter
\newcommand*{\hlink}[2][]{%
	\ifthenelse{\equal{#1}{}}%
	{\href{#2}{#2}}%
	{\href{#2}{#1}}}
\renewcommand*{\httplink}[2][]{%
	\ifthenelse{\equal{#1}{}}%
	{\href{https://#2}{#2}}%
	{\href{https://#2}{#1}}}
\makeatother

% redefine social region to support skype.
\makeatletter
\newcommand*{\skypesocialsymbol}{\faSkype~}
\RenewDocumentCommand{\social}{O{}O{}m}{%
	\ifthenelse{\equal{#2}{}}%
    {%
    	\ifthenelse{\equal{#1}{linkedin}}{\collectionadd[linkedin]{socials}{\protect\httplink[#3]{www.linkedin.com/in/#3}}}{}%
    	\ifthenelse{\equal{#1}{twitter}} {\collectionadd[twitter]{socials} {\protect\httplink[#3]{www.twitter.com/#3}}}    {}%
    	\ifthenelse{\equal{#1}{facebook}}{\collectionadd[facebook]{socials}{\protect\httplink[#3]{www.facebook.com/#3}}}   {}%
    	\ifthenelse{\equal{#1}{github}}  {\collectionadd[github]{socials}  {\protect\httplink[#3]{github.com/#3}}}     {}%
    	\ifthenelse{\equal{#1}{skype}}   {\collectionadd[skype]{socials}   {\protect\hlink[#3]{skype:#3}}}              {}%
    }
    {\collectionadd[#1]{socials}{\protect\httplink[#3]{#2}}}}
\makeatother

\makeatletter\newcommand\myname{\@firstname~\@lastname}\makeatother
\makeatletter\newcommand\myphone{\@phone}\makeatother
\makeatletter\newcommand\myemail{\@email}\makeatother
\makeatletter\newcommand\myhomepage{\@homepage}\makeatother

\name{Roland}{Bogosi}
\title{Curriculum Vit\ae}
\email{roland@rolisoft.net}
\homepage{rolisoft.net}
\social[linkedin]{RolandBogosi}
\social[skype]{RoliSoft}
\social[github]{RoliSoft}

%\address{Unit 7537, PO Box 6945, London}{W1A 6US}{UK}
%\phone[fixed]{+44~20~3411~1758}
%\phone[fixed]{+1~224~366~7654}
\phone[fixed]{$\raisebox{.28ex}{$\scriptstyle\textbf{\scriptsize +}$}$40~743~510~358}

\makeatletter\renewcommand*{\bibliographyitemlabel}{\@biblabel{\arabic{enumiv}}}\makeatother
\begin{document}
\makecvtitle

\section{Technical Skills}

	\cvline{Languages}{C\# \textit{(12 years)}, Java \textit{(11 years)}, C/C++ \textit{(10 years)}, Go, Python, PHP \textit{(14 years)}, JavaScript, SQL, VHDL \textit{(FPGA)}, CAPL \textit{(CAN)}}
	\cvline{Main Focus}{Web, mobile and desktop development; High availability; Low-level programming within constraints and optimization; High-level programming, design patterns and abstract concepts; Penetration testing; Reverse engineering; Research}

\section{Experience}

	\cventry{September 2016--Present}{Senior Software and Security Engineer}{Lateral Inc}{Târgu-Mure\cb{s}}{}{
	% nyas responsibilities + ACGI
	% private project pitched (security monitor)
	% leading technologies in breakthrough research fields
	% hybrid blockchain
	% aided with GDRP compliance
	Working with leading technologies in breakthrough research fields with high-profile clients; pitched projects for the internal incubator and lead their development.
	}

	\cventry{December 2013--December 2015}{Software Engineer}{BrightAlien Ltd}{London (Remote)}{}{Autonomously developed and was solely responsible for 9 websites plus an internal dashboard; installed, administered and proactively monitored various Linux and Windows-based cloud server instances; ensured stability, uptime and performance on a site with a peak of 15 million unique users.}

	\cventry{July 2014--August 2014}{Software Engineer Internship}{Lynx Solutions SRL}{Târgu-Mure\cb{s}}{}{Took leadership and oversaw fellow interns through their journey of implementing the internship project; wrote full application specification document and developed the documented Android-based application with UI accommodating both smartphones and tablets alike.}
	
\section{Education}
	
	\cventry{September 2012--July 2016}{B.Sc. (Hons.) Computer Engineering}{Sapientia EMTE, Faculty of Technical and Human Sciences}{Târgu-Mure\cb{s}}{}{Gave multiple presentations in various fields; attended both XIV. and XV. Scientific Students' Associations Conference with projects which both separately won awards; participated in four research groups as an extracurricular activity; graduated with a thesis in the field of Information Security, which received the maximal grade of 10.}

\section{Certifications}

	\cvitemwithcomment{}{\textbf{Offensive Security Certified Expert (OSCE)}, Offensive Security}{October 2018}
	\cvitemwithcomment{}{\textbf{Offensive Security Certified Professional (OSCP)}, Offensive Security}{May 2017}
	\cvitemwithcomment{}{\textbf{Security Specialist}, Certified Secure}{May 2016}
	\cvitemwithcomment{}{\textbf{Microcontrollers in Automotive Applications Specialist}, Continental AG}{January 2016}
	\cvitemwithcomment{}{\textbf{IPv6 Certified Sage}, Hurricane Electric}{March 2011}

\section{Awards, Honors \& Media Attention}

	\cvitemwithcomment{}{\textbf{Hack the Box "Best of the Best" Award}}{July 2017}
	\vspace{-2pt} {\small{Achieved and maintained 1\textsuperscript{st} place on the Hack the Box OSCP-like CTF game.}}

	\vspace{4pt}

	\cvitemwithcomment{}{\textbf{CERT-RO Coordinated Vulnerability Disclosure Program Pioneer}}{March 2017}
	\vspace{-2pt} {\small{Part of my personal project endeavors and academic research, I've autonomously analyzed various banking infrastructure and \textbf{found critical issues} in at least \textbf{two leading Romanian banks}, of the \textit{authentication and authorization bypass} level. I followed \textbf{responsible disclosure procedures}, notified the banks and the local CERT of these incidents. What followed was media attention in TV, radio, and online news, after \textbf{CERT-RO} published a congratulatory press release for being the first researcher to responsibly report to them: \httplink[https://goo.gl/D3w8b3]{cert.ro/citeste/divulgarea-coordonat-a-vulnerabilit-ilor-component-esen-ial-a-securit-ii-cibernetice}}}

	\vspace{4pt}

	\cvitemwithcomment{}{\textbf{XV. Scientific Students' Associations Conference Award}}{April 2016}
	\vspace{-2pt} {\small{1\textsuperscript{st} place for the "Black-Box Penetration Testing and Autonomous Vulnerability Assessment" research.}}

	\vspace{4pt}

	\cvitemwithcomment{}{\textbf{XIV. Scientific Students' Associations Conference Award}}{April 2015}
	\vspace{-2pt} {\small{Sponsors' choice prize for the "Sentiment Analysis of Social Networks" research.}}


\fancyhead[C]{\textcolor{color2}{\LARGE \textbf{\myname} | CV}}
\fancyfoot[L]{\textcolor{color2}{\phonesymbol \myphone {~~~{\rmfamily\textbullet}~~~} \emailsymbol\emaillink{\myemail} {~~~{\rmfamily\textbullet}~~~} \homepagesymbol\httplink{\myhomepage}}}


\section{Languages}

	\cvitem{Hungarian}{Native proficiency}
	\cvitem{English, Romanian}{Full professional proficiency}

\section{Publications}

	\cvitem{Deadbeat Disturbance Observer-based Rate Control in Wireless Networks for Moving Agents}{Lőrinc Márton, Tamás Vajda and Roland Bogosi; \textit{Research conducted under grant by Accenture Industrial Software Solutions pending publication.}}

	\cvitem{Linking Formal and Informal Structures Based on Faculty Members’ Email Communication Patterns}{Zoltán Kátai, Katalin Tünde Jánosi-Rancz and Roland Bogosi; \textit{Pending publication.}}

	\cvitem{Black-Box Penetration Testing and Autonomous Vulnerability Assessment}{Roland Bogosi; \textit{XV. Scientific Students' Associations Conference}, 2016.}

	\cvitem{Clustering in Mathematical Databases}{Szép Zoltán and Roland Bogosi; \textit{XV. Scientific Students' Associations Conference}, 2016.}

	\cvitem{Characterizing the Distance Between Formal and Informal Organizational Structures}{Katalin Tünde Jánosi-Rancz, Zoltán Kátai (and Roland Bogosi); \textit{MathInfo: International Conference in mathematics and Informatics}, 2015.}

	\cvitem{Sapiness Sentiment Analyser}{Katalin Tünde Jánosi-Rancz, Zoltán Kátai and Roland Bogosi; \textit{Acta Universitatis Sapientiae, Informatica,} vol. 7, no. 2, pp. 186--199, 2015.}

	\cvitem{Sentiment Analysis of Social Networks}{Roland Bogosi and Ozsváth Csilla; \textit{XIV. Scientific Students' Associations Conference}, 2015.}

\section{Notable Personal Projects}

	\subsection{Windows Subsystem for Linux Distribution Switcher}
	
	\begin{itemize}
		\item Project born out of sheer curiosity once WSL was announced, on whether I could swap the distribution shipped with the earliest version of WSL with something else, and then make a generic solution.

		\item A simple-to-use \textbf{Python} script was developed after I found a way, which was the first of its kind, and could swap in any distribution from an \textbf{ISO}, \textbf{rootfs tarball} or nearly any image on the \textbf{Docker Hub}.

		\item The repository has \textbf{1,300 stars} and 120 forks, as of August 2018. It has been featured in several tutorial sites, including \textbf{kali.org}'s initial way to get \textbf{Kali Linux in WSL} before the app store package, and \textbf{acknowledged by Microsoft} as a "cool community project": \httplink[https://goo.gl/GkVkAm]{www.kali.org/tutorials/kali-on-the-windows-subsystem-for-linux/}
	\end{itemize}

	\subsection{Host Scanner}

	\begin{itemize}
		\item An \textbf{open-source} application developed as an implementation companion to my bachelor's thesis. The purpose of the application is to perform \textbf{autonomous vulnerability assessment} using both active and passive scanning, or by analyzing earlier reports of 3$^{rd}$-party tools.

		\item The features and implementation techniques are unique to this application, a fact which was validated by an \textbf{award-winning} presentation at the \textbf{XV. Scientific Students' Associations Conference}.

		\item Project highlighted by several tutorial and new sites, including one of the supported data providers, \textbf{Mr. Looquer}, as a creative way to use their data.

		\item The application was developed in \textbf{C++} with strong \textbf{cross-platform} support with OS-specific implementations for \textit{Linux}, \textit{Windows} and \textit{BSD/Darwin} systems. The various helper scripts that come bundled with the application were developed in \textbf{Go}.
	\end{itemize}

	\subsection{RS TV Show Tracker}

	\begin{itemize}
		\item An \textbf{open-source} application which was born out of the need for the features it offered, as there were no alternative solutions for them at the time. To date, no software has so many features in this category, and as a result, under the four years it's been actively developed, its popularity has grown exponentially. For years, I single-handedly maintained it and pushed updates to hundreds of thousands of users on a monthly basis.

		\item The application was developed in \textbf{C\#} with an interface in \textbf{WPF}, and was kept constantly up-to-date with the newer technologies that have been released during its development phase.

		\item There are, as of May 2015, \textbf{135,700 active daily users} of the application, with the number of installations reaching into the \textbf{millions}. Development started in \textbf{February 2010} and halted in \textbf{January 2016} due to shifting priorities and avoidance of potential legal issues.
	\end{itemize}

	\subsection{BrightAlien Projects}

	\begin{itemize}
		\item \textbf{Full-stack development} of the project websites and continued maintenance, including initial \textbf{devops responsibilities} (such as cloud server deployment on DigitalOcean) and continued \textbf{proactive monitoring} and administration of the web, database and load-balancer servers.

		\item The websites were developed in \textbf{PHP} (\textbf{HHVM}) backed by \textbf{MySQL} servers in \textbf{multi-master replication} configuration and \textbf{ElasticSearch} instances for the search feature.

		\item At their peak, the sites had \textbf{15 million unique users} individually, whom were served by 4 geo-located and load-balanced front-end servers, attaining \textbf{100\% uptime} and \textbf{under 10 millisecond response times} for all users throughout peak times. Development started in \textbf{December 2013} and halted in \textbf{January 2016} due to shifting priorities to graduation.
	\end{itemize}

	\vspace{6pt}

	Full list of projects available on my GitHub profile and/or \httplink{lab.rolisoft.net/projects.html}

\section{Interests}

	\textbf{Self-teaching} is an important part of my lifestyle, I constantly \textbf{try out new languages}, \textbf{new technologies}, \textbf{new practices}, then go back and \textbf{re-do old projects}, with a twist, and I challenge myself to do it much better this time, by setting much higher goals.

\section{Early Personal Development}

	First exposure to the world of programming via PHP at the age of \textbf{11} followed by publishing the first dynamic webpage at the age of \textbf{12}. Published first \textbf{desktop application}, at the age of \textbf{13}, which utilized \textbf{databases}, \textbf{regular expressions}, \textbf{networking} and \textbf{screen-scraping}. Application was published to \textbf{Softpedia}, where it was accepted and reviewed by their staff, leading to a few hundred downloads during its lifetime. Received first income from \textbf{Google AdSense} at the age of \textbf{14}. Used \textbf{neural networks}, in order to train a software to \textbf{recognize} numbers and letters that appear on a webcam-provided image in \textbf{real-time}, at the age of \textbf{16}. Developed first \textbf{autonomous web-service} that relied on user-contributed data at the age of \textbf{17}, a \textbf{recommendation system}, which provides \textbf{tailored recommendations} based on a user's preference list.

	\vspace{6pt}

	Full timeline available at \httplink{rolisoft.net/\#education}

\end{document}